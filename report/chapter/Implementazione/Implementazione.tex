\documentclass[../../Report.tex]{subfiles}
\usepackage[italian]{babel}

\begin{document}
\chapter{Implementazione}
\section{Data Preprocessing}
La fase di preprocessing dei dati consiste nel preparare i dati per l'addestramento del modello.
Una volta acqusiti i dati e averne studiato le peculiarità e le caratteristiche, si procede con la pulizia dei dati.
Abbiamo usato il metodo \textit{pivot\_table} di Pandas per creare una tabella pivot che contiene i genome-score degli utenti per ogni film che lo possiede (andando così a ridurre la cardialità da 60k a 13.816).\\
In seguito a cià verrà effettuata una merge con un i generi dei film.
Ci siamo inoltre andati a focalizzare sui voti da parte degli utenti, raggruppandoli per film ci abbiamo calcolato la media dei voti.
Quest'ultimo capo sarà il nostro \textit{target} per l'addestramento del modello.\\

\section{Modeling}
La nostra fase di modellazione include uno studio di regressione basato sulla predizione del voto medio di un film dati i suoi genome-score (ed in seguito anche i generi).
Abbiamo inoltre applicato una tecnica di \textit{PCA} per ridurre la cardinalità dei dati, ma senza ottenere risultati soddisfacenti, quindi è stato deciso di non farne uso ulteriormente.\\

Eseguiamo il train-test split con un rapporto 80-20 ed in seguito addestreremo il nostro modello.
\subsection{Tecniche di ML Supervisionate con Approccio non-Deep}
In questa sezione verranno descritte le tecniche di ML supervisionate con approccio non-Deep utilizzate per la predizione del voto medio di un film.
\begin{itemize}
    \item \textbf{Linear Regression}: La regressione lineare è una tecnica di ML supervisionata che permette di predire il valore di una variabile dipendente a partire da una o più variabili indipendenti.
    \begin{itemize}
        \item \textbf{Ridge}: Ridge è un algoritmo di regressione lineare che utilizza un termine di regolarizzazione per eviare problemi di multicollinearità. Il parametro di regolarizzazione controlla l'importanza del termine di regolarizzazione nella funzione di costo
        \item \textbf{Lasso}: Lasso è un algoritmo di ML supervisionato che permette di effettuare la regressione di un dato mediante la creazione di un modello lineare.
    \end{itemize}
    \item \textbf{Random Forest}: Random Forest è un algoritmo di ML supervisionato che permette di effettuare la classificazione o la regressione di un dato mediante la creazione di più alberi (Metodo di Emsable Learning).
    \item \textbf{SVR}: Support Vector Regression è un algoritmo che cerca di trovare una funzione che minimizzi la distanza tra i dati di training e una fascia di tolleranza definita dall'utente.
    \item \textbf{KNN}: K-Nearest Neighbors è un algoritmo di ML supervisionato che permette di effettuare la classificazione o la regressione di un dato cerca di stimare il valore di una variabile dipendente su nuovi dati in base alla loro vicinanza ad altri dati di training.
    \item \textbf{NB Gaussian}: Naive Bayes è un algoritmo di ML supervisionato che permette di effettuare la classificazione o la regressione di un dato mediante la creazione di un modello probabilistico.
\end{itemize}

Per ogni modello precedentemente citato è stato applicato il Tuning dei parametri per cercare di ottimizzare il modello.
Attraverso la funzione \textit{RandomizedSearchCV} di Scikit-Learn abbiamo cercato di trovare i migliori parametri per ogni modello.

\subsection{Tecniche di ML Supervisionate con Reti Neurali}
\section*{Tabular Data}
\end{document}