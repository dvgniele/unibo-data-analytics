\documentclass[../../Report.tex]{subfiles}
\usepackage[italian]{babel}

\begin{document}
\chapter{Implementazione}
\section{Data Preprocessing}
La fase di preprocessing dei dati consiste nel preparare i dati per l'addestramento del modello.
Una volta acqusiti i dati e averne studiato le peculiaritá e le caratteristiche, si procede con la pulizia dei dati.

Abbiamo usato il metodo \textit{pivot\_table} di Pandas per creare una tabella pivot che contiene i genome-score degli utenti per ogni film che lo possiede(andando cosi a ridurre la cardialita' da 60k a 13.816).
In seguito a cio' verra' effettuata una merge con un i generi dei film applicando una One-Hot Encoding, attraverso la funzione \textit{get\_dummies()}.
Ci siamo inoltre andati a focalizzare sui voti da parte degli utenti, raggruppandoli per film ci abbiamo calcolato la media dei voti.

\section{Modeling}

\section*{Tabular Data}
\end{document}