\documentclass[../../Report.tex]{subfiles}
\usepackage[italian]{babel}

\begin{document}
\chapter{Introduzione}
L'obiettivo di questo report è presentare il progetto del corso di Data Analytics finalizzato alla predizione del voto medio di un film, date le sue caratteristiche, utilizzando un dataset proveniente da MovieLens \cite{movielens}, un recommendation system per contenuti video.
Il dataset contiene rating e tag per oltre 60.000 film, raccolti da più di 150.000 utenti negli anni 1995-2019. Ogni file del dataset dispone di un genoma che identifica una caratteristica del film e la sua rilevanza.

Per raggiungere questo obiettivo, abbiamo utilizzato tecniche di Machine Learning supervisionate tradizionali quali Linear Regression, SVM, NB in seguito verranno illustrate nel dettaglio; tecniche di ML basate su Reti Neurali ed infine modelli deep per Tabular Data.
In particolare, il report descrive il processo di acquisizione e preparazione dei dati, l'analisi esplorativa del dataset, la selezione delle feature rilevanti e la costruzione dei modelli predittivi.

Infine, il report conclude con una valutazione critica dei modelli creati, evidenziando i loro punti di forza e di debolezza, e suggerisce possibili sviluppi futuri per migliorare ulteriormente la predizione del voto medio dei film.

Lo studio è stato effettuato rispettando le componenti della pipeline studiata durante il corso, che è la seguente:
\begin{itemize}
    \item Data Acquisition
    \item Data Visualization
    \item Data Preprocessing
    \item Modeling (e tuning degli iperparametri)
    \item Test \& Evaluation
\end{itemize}

\end{document}