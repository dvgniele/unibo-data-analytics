\documentclass[../../Report.tex]{subfiles}
\usepackage[italian]{babel}

\begin{document}
\chapter{Metodologia}

\section{Data Acquisition}
In uno studio di Data Analysis, lo step di Data Acquisition è il primo della pipeline da seguire.
In questa fase abbiamo raccolto i dati necessari dal dataset \cite[MovieLens]{MovieLens}.
I file sono stati raccolti in una cartella denominata \texttt{ml-25m} contente un README.


\section{Dataset}
\label{dataset}

Movielens è un Recomendation System per contenuti multimediali, quali film, serie tv, documentari ecc.
Movielens mette a disposizione degli sviluppatori un dataset opensource, generator dal database TMDB (The Movie Database).
Questo datasedt contiene recensioni con voti e tag di oltre 60.000 film, raccolte da oltre 150.000 utenti durante il periodo che va dal 1995 fino al 2019.
\\
Il dataset è composto dai seguenti file:
\begin{itemize}
    \item \textit{genome-scores.csv}: contiene il \textit{relevance score} di ogni tag per tutti i film (ovvero, quanto un tag è importante per il dato film).
    \item \textit{genome-tags.csv}: contiene tutti i tag presenti all'interno del dataset.
    \item \textit{links.csv}: contiene gli ID di ogni film per i due database TMBD e IMDB.
    \item \textit{movies.csv}: contiene i titoli dei film, con i rispettivi generi.
    \item \textit{ratings.csv}: contiene più di 25.000.000 votazioni provenienti dalle recensioni degli utenti.
    \item \textit{tags.csv}: contiene i tag che sono stati assegnati ai film dagli utenti nelle rispettivi recensioni.
\end{itemize}

Per il nostro caso di studio, non abbiamo fatto uso di tutti i file disponibili, ma esclusivamente quelli che abbiamo considerato adatti per lo scopo.
Di seguito viene mostrata la struttura dei tali.
\\
Il dataset \textit{\textbf{ratings}} eèstato selezionato per poter far uso dei voti assegnati ai film dagli utenti, per poter quindi ottenere il voto mediio di tali film.
La struttura è la seguente
\begin{table}[H]
    \centering
    \begin{tabular}{|c|c|c|c|}
        \hline
        \textbf{userId} & \textbf{movieId} & \textbf{rating} & \textbf{timestamp} \\
        \hline
        1               & 296              & 5.0             & 1147880044         \\
        1               & 306              & 3.5             & 1147868817         \\
        1               & 307              & 5.0             & 1147868828         \\
        1               & 665              & 5.0             & 1147878820         \\
        1               & 899              & 3.5             & 1147868510         \\
        ...             & ...              & ...             & ...                \\
        \hline
    \end{tabular}
    \caption{ratings.csv}
    \label{tab:ratings_csv}
\end{table}

Il dataset \textit{\textbf{genome-scores}} è stato selezionato per poter identificare le relazioni che ci sono tra un dato voto ed i valori dei tag assegnati
\begin{table}[H]
    \centering
    \begin{tabular}{|c|c|c|}
        \hline
        \textbf{movieId} & \textbf{tagId} & \textbf{relevance}   \\
        \hline
        1                & 1              & 0.028749999999999998 \\
        1                & 2              & 0.023749999999999993 \\
        1                & 3              & 0.0625               \\
        1                & 4              & 0.07574999999999998  \\
        1                & 5              & 0.14075              \\
        ...              & ...            & ...                  \\
        \hline
    \end{tabular}
    \caption{genome-scores.csv}
    \label{tab:genome-scores_csv}
\end{table}



Infine, si è fatto uso del dataset \textit{\textbf{movies}} per ottenere le informazioni inerenti al genere di ogni film.
\begin{table}[H]
    \centering
    \begin{tabular}{|c|c|c|}
        \hline
        \textbf{movieId} & \textbf{title}           & \textbf{genres}              \\
        \hline
        1                & Toy Story (1995)         & Adventur$|$Animation$|$\dots \\
        2                & Jumanji (1995)           & Adventure$|$Children$|$\dots \\
        3                & Grumpier Old Men (1995)  & Comedy$|$Romance             \\
        4                & Waiting to Exhale (1995) & Comedy$|$Drama$|$\dots       \\
        \dots            & \dots                    & \dots                        \\
        \hline
    \end{tabular}
    \caption{movies.csv}
    \label{tab:movies.csv}
\end{table}


\section{Data Visualization}
Nella fase di \textit{data visualization} andremo a visualizzare alcune proprietà e peculiarità del dataset.

\subsubsection{Distribuzione dei ratings}
Nella seguente figura, è mostrata la distribuzione dei voti sul dataset.
\begin{figure}[H]
    \centering
    \includegraphics[width = .6\textwidth]{graph_distribution.png}
    \caption{Distribution of Ratings.}
    \label{fig:ratings_distribution_graph}
\end{figure}
Il gafico mostra la distribuzine dei vbti sul dataset, evidenziando un chiaro sbilanciamento nella distribuzione. In particolare, si può osservare una concentrzione significativamente maggiore di voti con valore di 3 o superiore rispetto ai voti inferiori.
Questo sbilanciamento è ulteriormente evidenziato dal fatto che i voti con valore di 3 o superiore sono 6 o più volte maggiori rispetto ai voti inferiori.
Tale distribuzione suggerisce una maggiore prevalenza di valutazioni positive rispetto a quelle negative o neutre nel dataset.
Questo risultato potrebbe essare utile per comprendere meglio le caratteristihe del dataset, ad esempio potrebbe suggerire la presenza di una tendenza positiva nelle valutazioni, o la necessità di bilancuiare meglio el categorie di voto.
\subsubsection{Relazione tra numero di voti e voto medio}
\begin{figure}[H]
    \centering
    \includegraphics[width = .6\textwidth]{realtionship_between_n_ratings_and_avg.png}
    \caption{Relationship between Number of Ratings and Average Rating.}
    \label{fig:realtionship_between_n_ratings_and_avg}
\end{figure}

\subsubsection{Voto medio per Genere}
\begin{figure}[H]
    \centering
    \includegraphics[width = .6\textwidth]{avg_by_genre.png}
    \caption{Average Ratings by Genre.}
    \label{fig:avg_by_genre}
\end{figure}


\section{Tabular Data}

\end{document}
